\begin{center}
A dissertation submitted to the University of Bristol in accordance with the requirements for the awards of degree of Doctor of Philosophy in the Faculty of Life Sciences 

\bigskip
\bigskip
\bigskip
\bigskip
\bigskip
\bigskip
\bigskip


Under the supervision of Professor Keith J. Edwards 


\bigskip


University of Bristol

Department of Life Sciences 

March 2021
\end{center}



\vfill
\hspace{0pt}

\setcounter{secnumdepth}{-8}
\pagenumbering{roman}
\cleardoublepage% important

\newpage

\section{Abstract}

Wheat's wide-ranging distribution, in addition to its vast levels of production and consumption make it an essential
component of global food security. Ever-increasing population sizes necessitate an increase in global wheat yields to
match. This thesis aims to contribute to this goal by addressing a broad range of seemingly disparate themes: evolution,
recombination and segregation distortion. What unites these themes is their methodological underpinnings - the use of
high-density genotyping arrays, which have undergone considerable development in the past decade.

The genetic diversity of wheat is limited by bottlenecks that have occurred in its evolutionary history, both through
polyploidization and domestication. This limitation presents difficulties for future yield increases, potentially
increasing wheat's susceptibility to pathogens. One area of interest is the rate of novel polymorphism formation over
time. The results presented here indicate that this question will be difficult to answer using molecular clock
methodology.

Another route to increasing wheat yield may be the manipulation of wheat recombination distribution, removing large
areas of linkage drag in the central regions of chromosomes. Previous work in barley suggests that an increase in
environmental temperature could shift recombination distribution inwards. The results presented here suggest that whilst
this might be the case for some chromosomes in wheat, for the majority of chromosomes, recombination distribution is
unaffected by changes in temperature.

Segregation distortion, a deviation from Mendelian ratios in progeny of a cross, is also investigated here, with a focus
on current practices of detection in the literature. My results indicate that many studies have been using inappropriate
methods for the detection of segregation distortion.

Also presented in this thesis are novel methods and tools for wheat research, such as the AutoCloner gene-cloning
pipeline, allowing researchers to efficiently clone large numbers of genes in previously unsequenced varieties.


\newpage

\section{Author's declaration}
I declare that the work in this dissertation was carried out in accordance with the requirements of the University's Regulations and Code of Practice for Research Degree Programmes and that it has not been submitted for any other academic award. Except where indicated by specific reference in the text, the work is the candidate's own work. Work done in collaboration with, or with the assistance of, others, is indicated as such. Any views expressed in the dissertation are those of the author.

\newpage


\section{Acknowledgements}

First and foremost I would like to thank my primary supervisor, Professor Keith Edwards for his unending support during
my PhD studies, whether that be through technical advice, moral support, providing opportunities to advance my own
scientific career through conferences and travel, or simply imparting wisdom acquired through his many years in science.
I would also like to thank my secondary supervisors, Gary Barker and Sacha Allen for their wisdom and support, whether
that be through general scientific advice, reading manuscripts, providing access to invaluable computing resources, or
organising travel and teaching opportunities.

This thesis would also not be possible without the many other members of the lab that provided invaluable knowledge and
help during the process - Amanda Burridge for her wealth of knowledge on molecular procedures and help in the lab, Paul
Wilkinson for advice and guidance on bioinformatic procedures, Mark Winfield for helpful theoretical discussions in the
office.

My family and friends have also been invaluable to me, in particular my father, mother and partner Swetha, who have
always been ready to lend an ear when I'm in need of a chat. I'm also thankful to the other members of the Biological
Sciences department, whether students or staff, who I've had the pleasure of sharing my time with. I'm always impressed
by the friendliness and warmth of the community, as well as the fascinating array of research in the department.

\newpage

\cleardoublepage% or \clearpage under the oneside option
\phantomsection% Mark a hyperref link location
